\chapter{Introdución}

El curso pasado (2016-2017) participé en el programa para emprendedores Yuzz “Jóvenes con ideas”, donde aprendí a convertir una idea empresarial en un modelo de negocio. 


Mi equipo promotor estaba compuesto por Bruno Rodríguez, psicólogo, y yo misma. Nuestra vocación hacia la aportación de soluciones en el campo de la salud mental, se vio impulsada por la adecuada combinación entre nuestras experiencias y estudios, dando lugar a nuestro proyecto empresarial Emozio. 


\section{Contexto}
Emozio será una plataforma web de búsqueda de psicólogos, donde el paciente cubre un test validado científicamente que permitirá a la plataforma asignar al profesional más adecuado para tratar su problema. Se trata pues de una plataforma de servicios de emparejamiento entre un paciente y el mejor profesional disponible. 


La finalidad es ofrecer el mejor especialista para solucionar el problema que tiene un determinado usuario, solventando la tediosa y difícil tarea de encontrar un psicólogo competente cuando surge la necesidad, y evitando terapias inefectivas. 


Por otro lado, la mayoría de los psicólogos que ejercen su profesión tienen la necesidad de encontrar un flujo constante de pacientes para mantener a flote su consulta/despacho, por lo que pertenecer a nuestro catálogo supondrá un plús añadido a sus servicios ya que éstos serán respaldados por un sello de calidad (Emozio) que hará posible el aumento y fidelización de clientes/pacientes. Así como, aportarles visualización y publicidad con costes más reducidos que los existentes.


\section{Objetivos}
El objetivo del proyecto es desarrollar una plataforma web que permita la búsqueda de psicólogos, donde, por un lado, los pacientes cubren un test que permitirá caracterizarlos, de forma que la plataforma pueda asignarles el profesional más adecuado para tratar su problema. Por otro lado, la plataforma ha de permitir a los psicólogos darse de alta como profesionales caracterizados por su \textit{expertise}. Dicha plataforma, ha de estar provista de diferentes servicios comunes como la gestión los usuarios (pacientes y psicólogos), posibilitar el filtrado de los psicólogos que aparecen como resultado del test en función de diferentes parámetros y poner en contacto el paciente con el psicólogo en cuestión... 


Se pretende que la plataforma desarrollada en el marco de trabajo del TFG, sirva como producto mínimo viable (PMV) del modelo de negocio, por lo que se persigue poder mostrar el producto como un prototipo a los usuarios finales. 


En concreto, se persigue conseguir los siguientes objetivos transversales:


\begin{itemize}
\item Desarrollar una plataforma web.
\item Permitir que los pacientes cubran un test que los pueda caracterizar.
\item Emparejar a los pacientes con los psicólogos que puedan tratarles.
\item Realizar la gestión de usuarios.
\item Permitir el filtrado en los resultados del test.
\item Permitir la comunicación entre pacientes y psicólogos.
\end{itemize}


\section{Estructura de la memoria}
