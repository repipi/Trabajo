\markboth{BIBLIOGRAFÍA}{BIBLIOGRAFÍA}
\addcontentsline{toc}{chapter}{Bibliografía}


\begin{thebibliography}{99}
%% EXEMPLO DE DOCUMENTO DESCARGADO DA WEB
%\bibitem{cuda} Nvidia CUDA programming guide. Versión 2.0, 2010. Dispoñible en {\it http://www.nvidia.com}.
%
%% EXEMPLO DE PÁXINA DA WIKIPEDIA
%\bibitem{cdma} Acceso múltiple por división de código. Artigo da wikipedia ({\it http://es.wikipedia.org}). Consultado o 2 de xaneiro do 2010.
%
%% EXEMEPLO DE LIBRO
%\bibitem{gonzalez} R.C. Gonzalez e R.E. Woods, {\it Digital image processing}, 3ª edición, Prentice Hall, New York, 2007.
%
%% EXEMPLO DE ARTIGO DE REVISTA
%\bibitem{patricia} P. González, J.C. Cartex e T.F. Pelas, ``Parallel computation of wavelet transforms using the lifting scheme'', {\it Journal of Supercomputing}, vol. 18, no. 4, pp. 141-152, junio 2001.

%%%%%%%%%%%%%%%%%%%%%%%%%%%%%%%%%%%%%%%%%%%%%%%%%%
%REALES
%%%%%%%%%%%%%%%%%%%%%%%%%%%%%%%%%%%%%%%%%%%%%%%%%%
% PMBOK
\bibitem{pmbok} PMBOK blablabla

% Pressman: Ingeniería del software. Un enfoque práctico
\bibitem{pressman} Pressman blablabla

\bibitem{sommerville} Sommerville blabla

\bibitem{github} GitHub  blaabla

\bibitem{mongodb_traspas} The magical marvels of MongoDB ({\it http://courseware.codeschool.com/the-magical-marvels-of-mongodb/the-magical-marvels-of-mongodb-slides.pdf}). Consultado o 2 de xaneiro do 2010.

\bibitem{mongodb_blog_mean_stack} Blog MongoDB. The Stack MEAN ({\it https://www.mongodb.com/blog/post/the-mean-stack-mongodb-expressjs-angularjs-and}). Consultado o 2 de xaneiro do 2010.

\bibitem{mongodb} MongoDB ({\it https://www.mongodb.com/es})

\bibitem{expressjs} ExpressJS ({\it http://expressjs.com/es/})

\bibitem{angularjs_arch} AngularJS Guide Architecture ({\it https://angular.io/guide/architecture})

\bibitem{angularjs_docs} AngularJS Docs ({\it https://angular.io/docs})

\bibitem{nodejs_about} NodeJS About ({\it https://nodejs.org/es/about/})

\bibitem{bootstrap_get} GetBootstrap ({\it http://getbootstrap.com/2.3.2/})

\bibitem{semanticui_github} Semantic UI GitHub ({\it https://github.com/Semantic-Org/Semantic-UI}) 

%%%%%%%%%%%%%%%%%%%%%%%%%%%%%%%%%%%%%%%%%%%%%%%%%%%%%%%%%%%%%%%%%%%%%%
%DISEÑO
%%%%%%%%%%%%%%%%%%%%%%%%%%%%%%%%%%%%%%%%%%%%%%%%%%%%%%%%%%%%%%%%%%%%%%

\bibitem{bootstrap_grid_basic} Bootstrap Grids. w3schools ({\it https://www.w3schools.com/bootstrap/bootstrap_grid_basic.asp}). Consultado de 29 de enero do 2017.

\bibitem{lato} Lato. Google Fonts ({\it https://fonts.google.com/specimen/Lato}). Consultado de 29 de enero do 2017.

\bibitem{merriweather} Merriweather. Google Fonts ({\it https://fonts.google.com/specimen/Merriweather}). Consultado de 29 de enero do 2017.

\bibitem{nordic_guidelines} Thorén C., {\it Nordic Guidelines for Computer Accessibility}, 2ª edición, Nordic Cooperation on Disability, 1998.

\bibitem{iso_ergonomics} ISO (2008). ISO 9241-171:2008 "Ergonomics of human-system interaction - Part 171: Guidance on software accessibility". Disponible en ({\it https://www.iso.org/standard/39080.html}). Consultado de 29 de enero do 2017. 

\bibitem{ekberg} Ekberg J. (2000). \textit{Un paso adelante: Diseño para todos}. Proyecto INCLUDE. CEAPAT-IMSERSO, Madrid.

\bibitem{aenor_req_acces} AENOR (2012a). UNE 139803:2012 "Requisitos de accesibilidad para contenidos en la Web". Disponible en (\it http://administracionelectronica.gob.es/pae_Home/pae_Estrategias/pae_Accesibilidad/pae_normativa/pae_eInclusion_Normas_Accesibilidad.html#.WBbsEYVwZZ0})

\bibitem{eval_heuris} Mª Paula González, Afra Pascual y Jesús Lorés.  Evaluación heurística. Universitat de Lleida. Dispoñible en {\it http://w.aipo.es/libro/pdf/15-Evaluacion-Heuristica.pdf}. Consultado de 29 de enero do 2017. 



\end{thebibliography}

