\chapter{Manual de instalación}

El proyecto ha sido desarrollado en Linux por lo que la instalación deberá hacerse a través de ese sistema operativo. 


Se deben seguir los siguientes pasos:

\begin{enumerate}
\item Se abre la terminal.
\item Sitúate sobre el directorio del proyecto \texttt{em\_web} con el comando \texttt{cd}.
\item \textbf{Instalación de la aplicación}
\begin{enumerate}
\item Se instala el gestor de paquetes NPM con el siguiente comando:
	\begin{lstlisting}[style=consola, numbers=none]
	$ sudo apt-get install npm
	\end{lstlisting}
\item Se instalan todas las dependencias de la aplicación que se encuentran en la carpeta local \texttt{node\_modules}, y que están listadas como dependencias en el archivo \texttt{package.json}:
	\begin{lstlisting}[style=consola, numbers=none]
	$ sudo npm install
	\end{lstlisting}
\end{enumerate}
\item \textbf{Instalación del gestor de bases de datos}
\begin{enumerate}
\item Se instala MongoDB con:
	\begin{lstlisting}[style=consola, numbers=none]
	$ sudo apt-get install -y mongodb-org
	\end{lstlisting}
\item Para correr MongoDB se utiliza:
	\begin{lstlisting}[style=consola, numbers=none]
	$ mongo
	\end{lstlisting}
\item Se crea la base de datos de Emozio con el comando que aparece a continuación. Una vez creada la base de datos, se puede acceder a ella con el mismo comando.
	\begin{lstlisting}[style=consola, numbers=none]
	$ use emozio
	\end{lstlisting}
\item Se cargan los datos en la base de datos ejecutando:
	\begin{lstlisting}[style=consola, numbers=none]
$ mongoimport -dab emozio -collection pacientes -type json -file bbdd_data/pacientes.json -jsonArray --drop
	
$ mongoimport -dab emozio -collection patologias -type json -file bbdd_data/patologias.json -jsonArray --drop
	
$ mongoimport -dab emozio -collection psicologos -type json -file bbdd_data/psicologos.json -jsonArray -drop
	
$ mongoimport -dab emozio -collection mensajes -type json -file bbdd_data/mensajes.json -jsonArray --drop
	\end{lstlisting}
\end{enumerate}
\end{enumerate}


\paragraph*{Modificación del proyecto}


Para modificar el proyecto, se tendrá que abrir con un editor de texto.
Si se desea utilizar el gestor de bases de datos Robomongo se deberá descargar a través de su página web: \url{https://robomongo.org/download} 

\paragraph*{Correr la aplicación}
\begin{enumerate}
\item Para correr la aplicación, se deberá ejecutar el comando:
	\begin{lstlisting}[style=consola, numbers=none]
	$ npm start
	\end{lstlisting}
\item Como nuestro servidor se encuentra escuchando el puerto 8000, para poder visualizar la aplicación en el navegador, se tendrá que indicar la dirección \texttt{localhost:8000}.
\end{enumerate}