\chapter{Conclusiones}

En el presente proyecto se ha conseguido desarrollar una plataforma web que sirve como producto mínimo viable a Emozio.


La plataforma permite que los pacientes que estén registrados puedan ser caracterizados cubriendo un test cuyas preguntas se corresponden con distintos síntomas que están presentes en las patologías de nuestra base de datos.


Una vez que el sistema determina el diagnóstico del paciente, es decir, qué patologías padece y en qué porcentaje, se le asocian aquellos psicólogos que tratan esas patologías. En cualquier momento, el psicólogo puede consultar el perfil público de estos psicólogos. En el perfil del psicólogo, el paciente puede dejarle un comentario con valoración.


El paciente puede decidir a cuál de los psicólogos (uno o varios) les envía una solicitud de cita. Una vez enviada, queda registrada en la página donde quedan listadas sus solicitudes pendientes de respuesta, aceptadas y rechazadas.


Cabe añadir, que el paciente puede registrarse, modificar sus datos o darse de baja.


Por otra parte, los psicólogos registrados en la plataforma pueden consultar las solicitudes que le han sido enviadas y aceptarlas o rechazarlas, dependiendo de su criterio. En estas solicitudes, se muestra el diagnóstico del paciente, por lo que puede entrar a valorar si puede o no tratarlo.


Los psicólogos, al aceptar una solicitud de cita, deben especificar cuándo se va a dar lugar. Esta cita, queda guardada en un calendario de citas al que puede consultar en cualquier momento.


Los psicólogos, además, pueden responder a los comentarios realizados por pacientes que aparezcan en su perfil.


Para poder pertenecer a la plataforma, puede rellenar un formulario que es enviado al correo electrónico de mi equipo Emozio. Nuestro equipo, entrará a valorar si el psicólogo es adecuado y cumple las certificaciones y garantías necesarias para pertenecer a nuestro catálogo de psicólogos. De la misma forma, cuando un psicólogo quiera modificar sus datos, también llegarán a nuestro correo para ser evaluados. Finalmente, un psicólogo puede darse de baja en cualquier momento.


\section{Posibles ampliaciones}
A raíz de haber realizado este proyecto, surgen nuevas oportunidades de mejora, arreglos y ampliaciones:
\begin{itemize}
\item Conseguir que la plataforma web sea completamente \textit{responsive}, es decir, que funcione corréctamente en cualquier tipo de dispositivo.
\item Conseguir garantizar que la plataforma cumpla todas las leyes vinculantes a la misma, como la Ley Orgánica 15/1999, del 13 de diciembre, sobre la Protección de Datos de Carácter Personal (LOPD).
\item Mejorar el algoritmo de emparejamiento mediante técnicas de inteligencia artificial tras el estudio e investigación de psicología que se realice sobre cómo caracterizar los pacientes, los psicólogos, las patologías y cómo se relacionan entre ambos.
\item Crear un perfil de administrador para que realice la evaluación de los candidatos psicólogos, haga de mediador entre pacientes y psicólogos si existe algún problema entre ambos; desde la plataforma.
\item Sistema de notificaciones cuando llegue una nueva solicitud (en el caso del perfil de psicólogo), o una respuesta a una solicitud (en el caso del perfil de paciente).
\item Permitir la cancelación de una cita por parte de psicólogos y pacientes.
\item Mejorar el nombre de los archivos y de las rutas de la aplicación para que sea más representativo; y a la larga, sea más fácil darle soporte.
\item Obtener los psicólogos que se han dado como resultado por orden de cercanía.
\item Añadir nuevos filtros en el filtrado de resultados.
\item Reflexionar y analizar una posible migración a una base de datos MySQL, ya que a pesar de que MongoDB de mucha flexibilidad, tiene demasiadas dependencias entre los documentos por lo que a veces los cambios no se consiguen reflejar en todo. Especialmente, a medida que crece el tamaño de la base de datos. 
\item Crear un historial de las citas que ha tenido un psicólogo con un determinado paciente, reflejando sus observaciones. De esta forma, podría tener un seguimiento del paciente siempre que quisiese a través de la plataforma.
\end{itemize}