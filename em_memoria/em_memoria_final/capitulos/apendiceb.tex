\chapter{Manual de instalación}

El proyecto ha sido desarrollado en Linux por lo que la instalación deberá hacerse a través de ese sistema operativo. 


Se deben seguir los siguientes pasos:

\begin{enumerate}
\item Se abre la terminal.
\item Sitúate sobre el directorio del proyecto \texttt{em_web} con el comando \texttt{cd}.
%\item \textbf{Instalación de la aplicación}
%\begin{enumerate}
%\item Se instala el gestor de paquetes NPM con el siguiente comando:
%%	\begin{listing}[style=consola, numbers=none]
%%	$ sudo apt-get install npm
%%	\end{listing}
%\item Se instalan todas las dependencias de la aplicación que se encuentran en la carpeta local \texttt{node_modules}, y que están listadas como dependencias en el archivo \texttt{package.json}:
%
%\end{enumerate}
%\item Instalación del gestor de bases de datos
%\begin{enumerate}
%\item Se instala MongoDB con:
%
%\item Para correr MongoDB se utiliza:
%
%\item Se crea la base de datos de Emozio con el comando que aparece a continuación. Una vez creada la base de datos, se puede acceder a ella con el mismo comando.
%
%\item Se cargan los datos en la base de datos ejecutando:
%
%\end{enumerate}
\end{enumerate}
